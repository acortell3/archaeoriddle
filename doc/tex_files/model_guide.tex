\documentclass{article}

% Useful packages
\usepackage[english]{babel}
\usepackage{framed}
\usepackage{graphicx}
\usepackage{algorithm, algorithmic, multicol}



\title{Rules for the model}
\author{CDAL}

\begin{document}
\maketitle

\section{Initial situation}
\subsection{Map}

\begin{figure}[htp]
    \centering
    \includegraphics[width=16cm]{01_map.png}
    \caption{Map of the world to work with.}
    \label{fig:map}
\end{figure}

\subsection{Population variables}

\subsubsection{hunter-gatherers}
\begin{itemize}
    \item $Pop_{hg}(t)$ is the total hunter-gatherer population
    \item $K_{hg}(t)$ is the total hunter-gatherer carrying capacity
    \item $Pop_{hgs}(t)$ is the hunter-gatherer population for each site
    \item $K_{hgf}(t)$ is the hunter-gatherer fission point for each site
    \item $G_{hg}(t)$ is the hunter-gatherer growth rate, extracted from a stochastic process (already designed).

\subsubsection{farmers}    
    \item $Pop_{f}(t)$ is the total farmer population
    \item $K_{f}(t)$ is the total farmer carrying capacity
    \item $Pop_{fs}(t)$ is the farmer population for each site
    \item $K_{ff}(t)$ is the farmer fission point for each site
    \item $G_{f}(t)$ is the farmer growth rate, extracted from a stochastic process (already designed).
\end{itemize}

\subsection{Movement}
for each $t$

Each group moves to different points in the map, and they stay in them, according to a double exponential distribution, with probability.

$$S_{hg} \sim L(\mu_{hg}, \lambda_{hg})$$
$$S_{f} \sim L(\mu_{f}, \lambda_{f})$$

Where $\mu$ is the mean distance that we consider feasible (to agree) and we could understand $\lambda$ as the variance, or how much do we accept that they settle out of that mean distance (to agree).

\subsection{Record generation and loss}
Already designed... too long to write in here, I'll explain it to you in person! :-)

\subsection{Initial conditions}

At $t(0)$ 
\begin{itemize}
    \item $Pop_{hg}(t_{0}) \approx K_{hg}(t_{0})$
    \item $Pop_{f}(t_{0}) << K_{f}(t_{0})$
    \item $G_{f}(t_{0}) > G_{hg}(t_{0})$
\end{itemize}

\section{Action}
For each $t$, \textbf{independently}

\begin{algorithm}[h]
  \begin{multicols}{2}
    \begin{algorithmic}
      \scriptsize
      \begin{flushleft}
      \textbf{Hunter-gatherers}\break
      1. Demographic growth (according to $G_{hg}(t)$\break
      
      2. Move (residential) \break 
      2.1 Stay with $P(y|S_{hg}$) \break
      
      3.\IF{$ Pop_{hgs}(t) > K_{hgf}(t) $}
        \STATE Exceeding people form (or go to) new site
        \ENDIF \break
      
      4. new sites increase chance of survival according to the density of their backing network (index, to think of)
        
      
      \end{flushleft}
    \end{algorithmic}
    \columnbreak
    \begin{algorithmic}
      \scriptsize
      \begin{flushleft}
      \textbf{Farmers}\break
      1. Demographic growth (according to $G_{f}(t)$\break
      
      2. Move (residential) \break 
      2.1 Stay with $P(y|S_{f}$) \break
      
      3.\IF{$ Pop_{fs}(t) > K_{ff}(t) $}
        \STATE Exceeding people form (or go to) new site
        \ENDIF \break
      
      4. new sites increase chance of survival according to the density of their backing network (index, to think of)
        
      \end{flushleft}
    \end{algorithmic}
  \end{multicols}
\end{algorithm}

\noindent For each $t$, \textbf{common action}

Decide if we want good or bad relationship between hunter-gatherers and farmers. For each time hunter-gatherers and farmers meet at the same spatial spot, if we want

\textbf{Good relationship}
Then, hunter-gatherers switch to farming according to

$$C \sim Bi(\theta, Pop_{hgs}(t))$$

And there is no additional population loss for any

\textbf{Bad relationship}
Then, hunter-gatherers and farmers' population decrease, at that spot, with

$$D_{hgs}(t) \sim Bi(\theta,Pop_{hgs}(t))$$
$$D_{fs}(t) \sim Bi(\theta,Pop_{fs}(t))$$

\section{RQ}

What is the impact of the hunter-gatherer attitude (bad relationship/good relationship) regarding the farmers' growth rate?

\end{document}
