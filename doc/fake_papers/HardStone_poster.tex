\documentclass[final]{beamer}

\definecolor{redcaa}{RGB}{152,0,34}

\usepackage[orientation=landscape,size=a2,scale=1.5,debug]{beamerposter}  % e.g. custom size poster
\title{On the relationship between Rabbitskinners and Poppychewers}
\author{Hardy Stones \& Pipelette Grand-Parlant Tenancier}
\institute{Human Language Technology and Pattern Recognition,RWTH Aachen University}
\date{April 2023}


% Set the font sizes for the headings and text
\setbeamerfont{title}{size=\huge}
\setbeamerfont{author}{size=\large}
\setbeamerfont{institute}{size=\large}
%\setbeamerfont{block title}{size=\Large}
%\setbeamerfont{block body}{size=\large}


% Remove the navigation symbols at the bottom of the poster
\setbeamertemplate{navigation symbols}{}

% Set the background color of the poster
\setbeamercolor{background canvas}{bg=white}
\setbeamercolor{block}{bg=white}
\setbeamercolor{structure}{fg=redcaa}

\setbeamercolor{block}{fg=black}
\setbeamercolor{block title}{fg=white,bg=redcaa}
\setbeamercolor{block body}{use=block title,bg=block title.fg}
\setbeamercolor{separation line}{bg=white,fg=redcaa}
\setbeamercolor{footline}{bg=redcaa,fg=white}
\setbeamercolor{footlinecolor}{fg=white,bg=redcaa}


\setbeamertemplate{footline}
{ 
    \begin{beamercolorbox}[wd=\paperwidth]{footlinecolor}
        \vskip5pt
        \begin{columns}
            \column{.05\paperwidth}
            \hspace{.2cm}
            \includegraphics[height=2cm]{caa2023.pdf}
            \column{.75\paperwidth}
            CAA | April 2023 -- 
            Hardy Stone \& Pipelette Grand-Parlant Tenancier
            \column{.05\paperwidth}
            \hfil
        \end{columns}
        \vspace{.1cm}
    \end{beamercolorbox} 
}

% Begin the document
\begin{document}

% Create the title block
\begin{frame}[t]
    \vspace{-.5cm}
    \begin{beamercolorbox}[wd=\paperwidth]{block title}
    \vspace{.8cm}
        \begin{columns}

            \column{.15\textwidth}

            \column{.8\textwidth}
            {
                \raggedleft
                \usebeamerfont{title}\textcolor{white}{\textbf{On the relationship between Rabbitskinners and Poppychewers}}\par
                \usebeamerfont{author}\textcolor{white}{Hardy Stone \& Pipelette Grand-Parlant Tenancier}\par
                \usebeamerfont{institute}\textcolor{black}{Institute of Normally Interesting \& Important Inquiries Carrying Evedicences}\par
            }

            \column{.05\textwidth}
            \includegraphics[height=4cm]{uni2.png}
        \end{columns}
    \vspace{.8cm}

    \end{beamercolorbox}


    \vspace{2cm}


    \begin{columns}[t]

        \column{.22\textwidth}
% Create the first block
        \begin{block}{\textbf{Introduction}}
            Rabbitskinners and Poppychewers were two distinct populations that inhabited the central and Western regions of Rabbithole during the Middle Prehistoric Period from 7400BP to 6500BP. Despite their radically different subsistence strategies, both groups coexisted in the region for a significant period, and there has been much debate among archaeologists about the nature of their interactions.
        \end{block}
        \vspace{3cm}

% Create the second block
    \begin{block}{\textbf{Methods}}
            The authors developed a specific quantitative method called the ``Test of Balanced Equilibrium'' (TBE) to demonstrate the relationship between Rabbitskinners and Poppychewers. The TBE equation is formulated as follows:

            \begin{equation}
                TBE = \sum_{i=1}^{n} \sum_{j=1}^{n} w_{ij} (N_i - N_j)
            \end{equation}

            In this equation, $N_i$ and $N_j$ represent the opulation sizes of two different settlements (or groups of settlements), and $w_{ij}$ is a weight that reflects the degree of interaction between these two settlements. This parameter will be fixed to 1 as the best tradeoff between complexity and accuracy.
        \end{block}

        \column{.45\textwidth}
% Create the third block
        \begin{block}{\textbf{Results}}
            The settlements were grouped into four different categories, and the TBE was calculated separately for each category. The results showed that Rabbitskinners and Poppychewers had very intense and peaceful interactions despite their different subsistence strategies. The maps of Rabbit hole during the Old age (~7200 - 7000 BP) and after Farmer extension (~7000 - 6800 BP) are shown in Figures 1 and 2, respectively.

            \begin{figure}
                \label{fig:twomaps}
                \centering
                \includegraphics[width=.4\textwidth]{all_gpe}
                \makebox[.9\textwidth][c]{
                    \includegraphics[width=.18\textwidth]{oldages}
                    \includegraphics[width=.18\textwidth]{newages}
                }
                \caption{The maps of Rabbithole during the Old age ( 7200 - 7000 BP) and after the Farmer extension ( 7000 - 6800 BP) are shown in Figures 1 and 2, respectively. First map show the distribution of all known sites from Rabbithole.}
            \end{figure}

            Our computed TBE is of approximatedly $1.5$ (after adjusting for specialist knowledge) ; which clearly show that both populations where peacefully interacting (as a reminder: when $TBE< -1$  relation are conflictual, when  $TBE = 0$  relation are neutral, when $TBE > 1$ relation are peaceful). This makes sense when looking at the reconstruction of the resources availability made by Stone~et~al.~(1966): given the complexity and uneven distribution of the resources populations needed to collaborate to survive. 

        \end{block}

% Create the fourth block
        \column{.22\textwidth}
    \begin{block}{\textbf{Conclusion}}
            The authors' TBE method provides a novel and powerful tool for studying the relationships between different populations in prehistoric times. The results of this study suggest that Rabbitskinners and Poppychewers had a more symbiotic relationship characterized by trade and exchange rather than competition over resources. This study contributes to a better understanding of the history of Rabbithole and the nature of interactions between its inhabitants.
        \end{block}

% Create the references block
        \begin{block}{\textbf{References}}
            Stone, H., \& Pants, F. (1966). Paleo-ecological reconstruction of Rabbithole environment during the between 8000BP to 6000BP. Journal of Ecological Research in Ancient Environments, 20(2), 45-56.
        \end{block}
% Create the references block
    \begin{block}{\textbf{Additional note}}
        \small
        This project is part of the Archaoriddle project, it is supported by the MSCA-IF grant no. 101020631 and the BA/Leverhulme grant SRG2223\textbackslash230262. If you want to help us understand what happened between Poppychewers and Rabbitskinners flash the QR-code below. You will have a chance to receive a £650 grant to join us at the EAA in Belfast! 
            \begin{figure}
                \includegraphics[width=.5\textwidth]{qrcode}
            \end{figure}
            {\tiny  All results shown here are derived from simulated dataset, don't reflect any real scenario and should not be take as valid scientific exploration.\\
            }
        \end{block}

    \end{columns}

\end{frame}
\end{document}


